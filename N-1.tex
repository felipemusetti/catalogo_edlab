\pagestyle{n-1}
\label{n-1}

\begin{textblock*}{5.625in}(0pt,0pt)%
\vspace*{-3.2cm}
\hspace*{-1.95cm}\includegraphics*[width=160mm]{./imgs/N-1.png}
\end{textblock*}

\pagebreak %RIOT PUSSY

\begin{center}
\hspace*{.5cm}\includegraphics[width=92mm]{./grid/riot.jpg}
\end{center}

\hspace*{-7cm}\hrulefill\hspace*{-7cm}

\medskip

\noindent{}{\slsc{Nós somos Pussy Riot}} é um combo e também um libelo que se interpõe entre o patriarcado capitalista e encarcerador e as mulheres que ele vitimiza, de mais de uma forma: seu conteúdo é revolucionário, mas também o são as várias etapas de seu processo produtivo.

Ele é composto por {\slsc{Riot days}}, um relato cru, alucinatório e apaixonado sobre a prisão e julgamento de Maria Alyokhina em uma colônia penal nos Urais após participar junto com Pussy Riot, sua banda punk feminista, de um protesto punk contra Putin em uma igreja ortodoxa.
E também de dois cordéis --- {\slsc{Sobre (Viver)}} e {\slsc{Engaiolaram"-nos}} --- tudo embalado em balaclavas coloridas, símbolo universal de resistência e luta social, confeccionadas especialmente pela Cooperativa Libertas, de mulheres egressas da máquina do sistema penitenciário brasileiro. Entre a banda Pussy Riot, perseguida em uma Rússia onde direitos das minorias são tolhidos, e as brasileiras que sofrem em um sistema desumano, há muito em comum. 

\vfill

\hspace*{-.4cm}\begin{minipage}[c]{.5\linewidth}
\small{
{\Formular{\textbf{
\hspace*{-.1cm}Título: Nós somos Pussy Riot\\
Autor: Maria Alyokhina\\ 
ISBN: 978-65-8109-713-4\\
Páginas: 216\\
Formato: 14x21cm\\
Preço: R\$ 89,90\\
Editora: n-1 \& Hedra\\
Disponibilidade: Disponível
}}}}
\end{minipage}

\pagebreak %CORPOS QUE IMPORTAM, JUDITH BUTLER


\begin{center}
\hspace*{-2.5cm}\raisebox{6.8cm}{\rotatebox[origin=t]{90}{\huge\Formular{\textbf{Lançamento}}}}
\hspace*{2.5cm}\includegraphics[width=92mm]{./grid/butler.png}
\end{center}

\hspace*{-7cm}\hrulefill\hspace*{-7cm}

\medskip

\noindent{}{\slsc{Corpos que importam}} é sobre pensar a hegemonia heterossexual na criação de matérias [{\slsc{matter}}] sexuais e políticas. Quais são as limitações pelas quais corpos são materializados como “sexuados”? E como devemos entender a “questão” [{\slsc{matter}}] do sexo e dos corpos de modo mais geral, como a circunscrição repetida e violenta da inteligibilidade cultural? Quais corpos importarão [{\slsc{matter}}] --- e por quê?

Nas palavras da filósofa pós"-estruturalista Judith Butler: “Ofereço este texto em parte como forma de reconsiderar algumas seções de meu livro {\slsc{Problemas de gênero}} que causaram confusão, mas também como um esforço para pensar mais sobre o funcionamento da hegemonia heterossexual na criação de matérias [{\slsc{matter}}] sexuais e políticas. Como uma rearticulação crítica de práticas teóricas, incluindo estudos feministas e {\slsc{queer}}, esta obra não pretende ser programática. E ainda como tentativa de esclarecer minhas “intenções”, ela também parece destinada a produzir novos conjuntos de mal"-entendidos. Espero que, ao menos, eles se provem produtivos.”


\vfill

\hspace*{-.4cm}\begin{minipage}[c]{.5\linewidth}
\small{
{\Formular{\textbf{
\hspace*{-.1cm}Título: Corpos que importam – \\os limites discursivos do “sexo”\\
Autor: Judith Butler\\ 
ISBN: 978-65-8109-704-2\\
Páginas: 400\\
Formato: 14x21cm\\
Preço: R\$ 98,00\\
Editora: n-1 \& Crocodilo\\
Disponibilidade: Disponível
}}}}
\end{minipage}

\pagebreak %SOMOS NOSSO CÉREBRO?

\begin{center}
\hspace*{.5cm}\includegraphics[width=92mm]{./grid/cerebro.png}
\end{center}

\hspace*{-7cm}\hrulefill\hspace*{-7cm}

\medskip

\noindent{}Explorando o neurocentrismo, a crença de que “somos nossos cérebros” se difundiu nos anos 1990. Encorajados pelos avanços da neuroimagem, as humanidades e as ciências sociais adotaram uma “virada neurocientífica” na forma de neuro"-subespecialidades em campos como antropologia, estética, educação, história, direito, sociologia e teologia. Empresas comerciais duvidosas, mas bem"-sucedidas, como “neuromarketing” e “neurobica” surgiram para tirar proveito da sensibilidade aumentada para todo o universo neuro.

Embora não seja hegemônica nem monolítica, a visão neurocêntrica encarna uma poderosa ideologia que está no cerne de alguns dos mais importantes debates filosóficos, éticos, científicos e políticos da atualidade. {\slsc{Somos nosso cérebro?}} escolhido livro do ano em 2018 pela {\slsc{International Society for the History of the Neurosciences}}, examina a lógica interna de tal ideologia, sua genealogia e suas principais encarnações contemporâneas. 


\vfill

\hspace*{-.4cm}\begin{minipage}[c]{.5\linewidth}
\small{
{\Formular{\textbf{
\hspace*{-.1cm}Título: Somos nosso cérebro?\\ Neurociências, subjetividade, cultura\\
Autor: Francisco Ortega e Fernando vidal\\ 
ISBN: 978-65-9582-035-7\\
Páginas: 346\\
Formato: 16x23cm\\
Preço: R\$ 80,00\\
Editora: n-1 \& Hedra\\
Disponibilidade: Disponível
}}}}
\end{minipage}

\pagebreak %PRAGMATISMO PULSIONAL


\begin{center}
\hspace*{-2.5cm}\raisebox{6.8cm}{\rotatebox[origin=t]{90}{\huge\Formular{\textbf{Lançamento}}}}
\hspace*{2.5cm}\includegraphics[width=92mm]{./grid/gozo.jpg}
\end{center}

\hspace*{-7cm}\hrulefill\hspace*{-7cm}

\medskip

\noindent{}Raros são os psicanalistas que têm tamanho domínio de Lacan e de Guattari a um só tempo, e que a exemplo de João Perci Schiavon, professor da \scalebox{.8}{PUC-SP} e autor de obras como {\slsc{A lógica da vida desejante}}, conseguem alcance clínico e filosófico de tal envergadura.  

A uma atmosfera de convite a experimentação do fluxo inconsciente e aposta clínica num diapasão em tudo singular entrecruza assuntos --- a contribuição freudiana à linhagem filosófica que vai de Nietzsche a Deleuze. E a pulsão reaparece sob nova luz: o gozo, este que não serve a nenhum bem, engendra todos os bens possíveis e todas as utilidades, algo desconcertante se tivermos em vista o título do escrito, {\slsc{Pragmatismo pulsional}}.

O livro pode ser lido e vivido como “experiência” subjetiva, clínica, filosófica, micropolítica --- isto é, de transformação. O que mais se pode exigir de um livro hoje além de que faça diferença, e diferença para a vida? A pulsão é uma autoridade no que diz respeito ao vivo ou ao desejo. Ela só precisa ser exercida, e o nome desse exercício é cura.

\vfill

\hspace*{-.4cm}\begin{minipage}[c]{.5\linewidth}
\small{
{\Formular{\textbf{
\hspace*{-.1cm}Título: Pragmatismo pulsional –\\ clínica psicanalítica\\
Autor: João Perci Schiavon\\ 
ISBN: 978-65-8109-706-6\\
Páginas: 336\\
Formato: 14x21cm\\
Preço: R\$ 66,00\\
Editora: n-1\\
Disponibilidade: Disponível
}}}}
\end{minipage}

\pagebreak %FASCISMO OU REVOLUÇÃO? MAURICIO LAZZARATO


\begin{center}
\hspace*{.5cm}\includegraphics[width=92mm]{./grid/lazzarato.png}
\end{center}

\hspace*{-7cm}\hrulefill\hspace*{-7cm}

\medskip

\noindent{}O fascismo histórico foi tão moderno quanto o capitalismo (já que é uma de suas expressões), como demonstrado nitidamente pelo futurismo italiano. O mesmo ocorre com o novo fascismo, que é um ciberfascismo. Ele põe em xeque todas as utopias tecnociber (do ciberpunk ao ciberfeminismo, da ciberesfera à cibercultura) que desde o pós"-guerra, principalmente a partir dos anos 1970, viam nas máquinas cibernéticas uma dupla promessa: a de uma nova subjetividade pós"-humana e a da liberação da dominação capitalista, cuja ruptura viria das máquinas e não da política. Das revoluções da técnica e não da organização revolucionária.

Bolsonaro e Trump utilizaram todas as tecnologias disponíveis da comunicação digital, mas suas vitórias não vêm da tecnologia. São o resultado da máquina política e estratégia que agencia uma micropolítica dos afetos tristes (frustração, ódio, inveja, angústia medo) com relação à macropolítica de um novo fascismo, que dá consistência política às subjetividades devastadas na financeirização.

\vfill

\hspace*{-.4cm}\begin{minipage}[c]{.5\linewidth}
\small{
{\Formular{\textbf{
\hspace*{-.1cm}Título: Fascimo ou Revolução?\\
Autor: Mauricio Lazzarato\\ 
ISBN: 978-85-6694-381-8\\
Páginas: 208\\
Formato: 14x21cm\\
Preço: R\$ 69,00\\
Editora: n-1\\
Disponibilidade: Disponível
}}}}
\end{minipage}

\pagebreak %ESPAÇO CORUJA

\begin{center}
\hspace*{-2.5cm}\raisebox{6.8cm}{\rotatebox[origin=t]{90}{\huge\Formular{\textbf{Lançamento}}}}
\hspace*{2.5cm}\includegraphics[width=92mm]{./grid/coruja.jpg}
\end{center}

\hspace*{-7cm}\hrulefill\hspace*{-7cm}

\medskip

\noindent{}“A formulação do Projeto de Lei do Espaço Coruja foi uma das ações de maior impacto emocional para Marielle. Era a sua história: uma mulher negra, favelada, que foi mãe jovem e precisava de um lugar seguro para deixar sua filha enquanto trabalhava e estudava. Ela teve uma rede de familiares e de amigos que lhe garantiu avançar e concluir seus estudos. Mas conhecia as diversas e duras realidades da maioria das mulheres que não possuem essa oportunidade. Com empatia e olhar solidário, ela propôs, como legisladora, a criação do Espaço Infantil Noturno. Foram muitos os momentos em que a vi preocupada e empenhada com a elaboração desse projeto tão caro a ela. Ouvinte ativa às críticas, sempre buscou ponderar, mas tinha em seu próprio corpo e história de vida a certeza da urgência. Com essa lei, crianças terão direito a um lugar seguro enquanto suas mães e pais trabalham e/ou estudam --- em um mundo mais justo e igualitário e movimentando as estruturas dessa sociedade ainda tão patriarcal, misógina, machista e racista.” [Mônica Benício]


\vfill

\hspace*{-.4cm}\begin{minipage}[c]{.5\linewidth}
\small{
{\Formular{\textbf{
\hspace*{-.1cm}Título: Espaço Coruja\\
Autor: Amanda Mendonça\\ e Pâmella Passos\\ 
ISBN: 978-65-8109-705-9\\
Páginas: 64\\
Formato: 10x16cm\\
Preço: R\$ 20,00\\
Editora: n-1\\
Disponibilidade: Disponível
}}}}
\end{minipage}

\pagebreak %UM PIANO NAS BARRICADAS, TARÌ

\begin{center}
\hspace*{.5cm}\includegraphics[width=92mm]{./grid/barricada.png}
\end{center}

\hspace*{-7cm}\hrulefill\hspace*{-7cm}

\medskip

\noindent{}Nos filmes sociais e políticos dos anos 1970 na Itália, a Autonomia é apresentada como um método intermediário entre Marx e a antipsiquiatria, a Comuna de Paris e a contracultura, o dadaísmo e o insurrecionalismo, o operaísmo e o feminismo e muitas coisas com outras coisas. Mas apresentou acima de tudo descontinuidade profunda com a prática do Movimento Operário oficial. Não era e nunca foi uma organização, mas uma multiplicidade que se organizava a partir de onde residiam, trabalhavam ou estudavam os sujeitos que a deram forma.

Na Autonomia, muitas autonomias específicas surgiram e coexistiram: operários, estudantes, mulheres, homossexuais, prisioneiros --- ou de qualquer um que escolhesse, a partir de suas próprias contradições, o caminho da luta por um modo reluzente de subversão da vida contra o trabalho assalariado e o Estado. Se o Movimento dos anos 1970 acabou sucumbindo às forças combinadas da máquina estatal e do Partido Comunista, a história da autonomia destaca"-se desse contexto, pois é a de uma aventura revolucionária cuja incandescência atual é mais relevante do que nunca. 

\vfill

\hspace*{-.4cm}\begin{minipage}[c]{.5\linewidth}
\small{
{\Formular{\textbf{
\hspace*{-.1cm}Título: Um piano nas barricadas –\\ por uma história da autonomia\\
Autor: Marcelo Tarì\\ 
ISBN: 978-65-8042-104-6\\
Páginas: 384\\
Formato: 12x19cm\\
Preço: R\$ 80,00\\
Editora: n-1 \& Glac\\
Disponibilidade: Disponível
}}}}
\end{minipage}

\pagebreak
\pagestyle{n-1cat}

\begin{multicols}{2}
\begin{enumerate}
\nohyphens{
\item Pandemia - Rexistir, {\Formular{\textbf{Eduardo Viveiros de Castro; Achille Mbembe; Carmem Silva; Antonio Negri; Cristina Ribas; Jean Tible; Tatiana Nascimento; Tiqqun; Eduardo Passos; Danichi Hausen Mizoguchi; Yuk Hui}}}
\item Potências do tempo, {\Formular{\textbf{David Lapoujade}}}
\item Declaração, {\Formular{\textbf{Antonio Negri; Michael Hardt}}}
\item Manifesto contrassexual, {\Formular{\textbf{Beatriz Preciado}}}
\item O aracniano e outros textos, {\Formular{\textbf{Fernand Deligny}}}
\item Deleuze, os movimentos aberrantes, {\Formular{\textbf{David Lapoujade}}}
\item Aos nossos amigos, {\Formular{\textbf{Comitê Invisível}}}
\item Teoria King Kong, {\Formular{\textbf{Virginie Despentes}}}
\item Guattari, {\Formular{\textbf{Kuniichi Uno; Laymert Garcia dos Santos}}}
\item Quando e como eu li Foucault, {\Formular{\textbf{Antonio Negri}}}
\item O avesso do niilismo, {\Formular{\textbf{Peter Pál Pelbart}}}
\item A missão, {\Formular{\textbf{Heiner Müller}}}
\item William James, a construção da experiência, {\Formular{\textbf{David Lapoujade}}}
\item Nietzsche - O bufão dos deuses, {\Formular{\textbf{Maria Cristina Franco Ferraz}}}
\item Impressões de Michel Foucault, {\Formular{\textbf{Roberto Machado}}}
\item Fabulações do corpo japonês, {\Formular{\textbf{Christine Greiner}}}
\item As existências mínimas, {\Formular{\textbf{David Lapoujade}}}
\item Hegel e o Haiti, {\Formular{\textbf{Susan Buck-Morss}}}
\item Brazuca, negão e sebento, {\Formular{\textbf{Jean-Christophe Goddard}}}
\item Motim e destituição agora, {\Formular{\textbf{Comitê Invisível}}}
\item Crítica da razão negra, {\Formular{\textbf{Achille Mbembe}}}
\item Testo junkie, {\Formular{\textbf{Paul B. Preciado}}}
\item O universo inacabado, {\Formular{\textbf{Mario Novello}}}
\item Cartas e outros textos, {\Formular{\textbf{Gilles Deleuze}}}
\item Nietzsche e a filosofia, {\Formular{\textbf{Gilles Deleuze}}}
\item Hijikata tatsumi, {\Formular{\textbf{Kuniichi Uno}}}
\item Spartakus, {\Formular{\textbf{Furio Jesi}}}
\item Agamben, {\Formular{\textbf{Giacoia Jr., Oswaldo}}}
\item UPP - A redução da favela em três letras, {\Formular{\textbf{Marielle Franco}}}
\item Cinco dias em março, {\Formular{\textbf{Toshiki Okada}}}
\item Os vagabundos eficazes, {\Formular{\textbf{Fernand Deligny}}}
\item O enigma da revolta, {\Formular{\textbf{Michel Foucault}}}
\item Arqueofeminismo
\item Contribuição para a guerra em curso, {\Formular{\textbf{Tiqqun}}}
\item Ética bixa, {\Formular{\textbf{Paco Vidarte}}}
\item Ensaios do assombro, {\Formular{\textbf{Peter Pál Pelbart}}}
\item Metafísicas canibais, {\Formular{\textbf{Eduardo Viveiros de Castro}}}
\item O governo do homem endividado, {\Formular{\textbf{Maurizio Lazzarato}}}
\item Leituras do corpo no Japão, {\Formular{\textbf{Christine Greiner}}}
\item Pragmatismo pulsional, {\Formular{\textbf{João Perci Schiavon}}}
\item Ruptura, {\Formular{\textbf{Centelha}}}
\item Às voltas com Lautréamont, {\Formular{\textbf{Laymert Garcia dos Santos}}}
\item Afrotopia, {\Formular{\textbf{Felwine Sarr}}}
\item Fascismo ou revolução?, {\Formular{\textbf{Maurizio Lazzarato}}}
\item Corpos que importam, {\Formular{\textbf{Judith Butler}}}
\item Somos nosso cérebro?, {\Formular{\textbf{Francisco Ortega; Fernando Vidal}}}
\item Ritornelos, {\Formular{\textbf{Félix Guattari}}}
\item Contracultura, entre a curtição e o experimental, {\Formular{\textbf{Celso Favaretto}}}
}
\end{enumerate}
\end{multicols}

\pagebreak